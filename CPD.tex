% Please do not change the document class
\documentclass{scrartcl}

% Please do not change these packages
\usepackage[hidelinks]{hyperref}
\usepackage[none]{hyphenat}
\usepackage{setspace}
\doublespace

% You may add additional packages here
\usepackage{amsmath}

% Please include a clear, concise, and descriptive title
\title{Continuing Personal Development}

% Please do not change the subtitle
\subtitle{COMP240 - CPD Report}

% Please put your student number in the author field
\author{1702208}

\begin{document}

\maketitle

\section{Introduction}
Unfortunately, I have encountered problems across all my modules where I had to work both individually and collaboratively. However, this might also mean I have more space to grow and develop myself in a more professional manner and be better prepared for the Game Industry.

\section{Interpersonal Domain - Providing Feedback}
Communication was always my worst skill I have so far. Due to my previous experience in collaborative work, I have very poorly developed such skill and therefore could not properly communicate with my team during this year. This had also affected the team-play I had in the team in this term. So, this is what I mostly need to develop. However, this is way too broad and not achievable in the short-term. So instead I want to focus on one specific idea - bringing up one constructive feedback as well as some personal ideas to improve the game during each team meeting. Therefore, I can let my team know about my ideas on the current stage of the game more clearly as well as making sure what I can and cannot do at the time.

\subsection{Action}
During each team meeting, I will bring at least one constructive feedback on the game to give the team a better understanding of what I feel like could be useful to see in the game.

\section{Dispositional Domain - Setting Up Working Times}
Equally balancing my work between all assignments and personal time is extremely hard for me. I find myself distracted by games very easily and therefore do much less work than I was planning to in the first place. I'm very aware of this negative habit of mine, but can rarely do anything about it on my own. However, small changes to planning might help to prevent this over time. 

Before, I was making a calendar, where I put work times. By doing this, I tried to make myself do some work each day for different assignments. Unfortunately, I was still distracted by other things and barely did any work during those times. This has probably been happening because I would either forget that I have planned it or didn't notice a notification about it on my phone. In either way, I should consider something that will make me remember to check my calendar or the tasks I have planned to do.

\subsection{Action}
Besides a calendar with working times, I will also have both a Trello board for big tasks I want to complete and sticky notes to remind me of smaller tasks that will hang close to my working place so that I will always see and remember the work I need to do.

\section{Affective Domain - Recording Emotions and Mental States}
While I'm personally am stable with my emotions and mental states at most times, the ability to describe these mental states during both individual and team projects is very important. In such a case, one can better understand himself and therefore find out whether these mental states can influence the overall outcome of a project or workflow. However, describing these states can be very difficult for me due to the little knowledge and understanding of my own emotions.

Nevertheless, it can be an important skill I could develop for myself - to try and describe my mental states and emotions during the development processes I take part in. In which case, if I find negative influence during some states of mine, I could immediately fight it myself if it's something light-weighted or look for help if necessary.

\subsection{Action}
During next year's term, I will record my emotions and mental states in the weekly reports.

\section{Cognitive Domain - Learning Programming Patterns}
During this term, I have truly learned a lot while working on the game as well as on personal assignments. I feel much more comfortable with programming tasks, different languages and debugging. However, everything I do is not always high-performance. This includes some code duplication and high memory usage. In which case, I will need to learn programming patterns to better understand how I can prevent these issues in the future.

This could help me with my workflow so that I don't spend extra time making everything clean and better in performance. Also, it would surely make me more employable for software related jobs, which is always a benefit.

\subsection{Action}
At best, each week I will read one chapter of programming patterns on gameprogrammingpatterns.com to get some grip of it.

\section{Procedural Domain - Splitting Tasks}
As I have mentioned before, I have learned a lot during this semester. But there is one simple thing that I think would greatly improve my workflow. I always assign my tasks very vaguely, which, in return takes a lot of time to complete. This not only makes it a bit harder to concentrate on one small task but also feels as if I'm barely doing anything. This especially is noticeable in the team project's development process.

Whenever I have partially completed one task, I would show it to my team, but due to the task being so vague, I cannot say it's done. Therefore, the team would think that the task is taking too long for me to complete, which badly influences me.

\subsection{Action}
During any planning, I will write down my personal tasks and split them down into much smaller ones, that I will be confident in completing within a day.

\section{Conclusion}
It has been a much more enjoyable term than the previous one due to my major improvement in many areas, such as programming and communication. These CPDs have greatly helped me reflect upon my weaknesses and overcome them. I'm hoping I will keep developing myself during this summer, so I return with fresh-new energy to work on a new project with a new team.

\bibliographystyle{ieeetran}
\bibliography{references}

\end{document}
