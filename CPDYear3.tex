% Please do not change the document class
\documentclass{scrartcl}

% Please do not change these packages
\usepackage[hidelinks]{hyperref}
\usepackage[none]{hyphenat}
\usepackage{setspace}
\doublespace

% You may add additional packages here
\usepackage{amsmath}

% Please include a clear, concise, and descriptive title
\title{Personal Evaluation and Development} % TO BE CHANGED

% Please do not change the subtitle
\subtitle{GAM320 - Evaluation}

% Please put your student number in the author field
\author{1702208}

\begin{document}

\maketitle

\section{Introduction}
Last year was a disaster for me, however, I'm very happy I've been invited to this team. The team is extremely nice and positive about the game and it's potential. I feel that we made a strong connection with each other. We can easily talk about the game, future ideas, mechanics and just have a chat to get to know each other better. However, there's never only a positive outcome in interacting with other people. There will always be some conflicts and misunderstandings, which will impact our production in the future. Luckily, it has not been crucial at this point, but I feel like the temptation is growing in all of us bit by bit. This has to be handled as soon as possible to prevent big losses for the whole team. That's what I will be trying to plan in this report - criticizing myself and my team for the good and looking for solutions for particular problems that may occur in the second semester.

\section{Prioritizing work}
Most of my time in the studio I've been working on our main prototype - a real-time strategy (RTS) game. Implementing basic mechanics, features. However, I've been doing that selfishly - without prioritizing my tasks in regards to minimal viable product (MVP). I was working on whatever I thought could be used in the game without asking our designers or looking into the design documents they made or anything I was interested in. It happened mostly because our designers were working on two prototypes at the same type, this one is a paper prototype. We had a basic idea of what we need in a digital prototype, however, we weren't told exactly what features to implement, which turned out to be us, programmers, doing whatever we feel is necessary. This did not turn out overwhelmingly terrible, however, I still lost some time working on unnecessary stuff that has not been considered yet. 

By this point, we could've had enough content for our MVP and start expanding the game by adding new features or mechanics, so that the game doesn't feel empty. This was my omission and I would want to avoid that in the next semester as well as this semester by analyzing my tasks more in-depth to ascertain my work is first of all beneficial to the project. This will help prioritize the features and mechanics of the game and get important things done quicker due to better insight of its importance and better knowledge of the detail of the task.

\section{Agile Development}
During our production, we are using agile development, which has to include every team member's skill sets, ideas, and preferences about the game. This has been relatively straightforward for our team. We have decided on several ideas at the end of our second year and had a rough idea of what we could do over the Summer. Unfortunately only a few of us did some work over Summer break, which was expected. However, we have removed one of the prototypes and instead concentrated on the other two. Due to designers working on different prototypes and only one being available to digitally prototype, I and another fellow programmer worked on the same prototype, which then became the main project that we unanimously decided on. Everyone's been enjoying the process and were motivated about its outcome. However, after a meeting with another team, several members of our team decided to throw away this prototype and pitch other ideas to each other that we potentially could start working on. 

Having worked on it for five weeks and then to throw it away was rather disappointing. Luckily, during the meeting, I've come up with the idea that everyone liked and didn't require throwing away our five-week-long work. This has most likely happened due to the team not being communicative enough to solve such problems that require only one specific discipline. To avoid this issue in the future I want to closely be in touch with designers to not only help them with any occurred problems but also have a better insight into my tasks as a programmer. This will also contribute to my other personal development that I talk about in the next section.

\section{Leading programming tasks}
I have never considered myself a good programmer, yet I always do my best to improve and find something I enjoy most in programming. This third year is the best opportunity to learn the most and enjoy development. However, other responsibilities hit me this time. Due to only having a lead artist in the team, I proclaimed myself as a lead programmer for this project. It gives me more chances to make mistakes that I could learn from as well as important aspects of such a role. Nevertheless, this is not without a reason.

After finding the game's unique selling point (USP) I've been talking about in the previous section, this role has hit me even harder. Due to presenting this USP, designers started showing me their progress in developing the idea visually and asking me about my thoughts and how I visualized it originally. Even though this is not a lead programmer's role to guide designers due to that they, as well as another programmer, started updating me more often and making sure I know what's going on so that I could lead the progress or fix any mistakes, if possible. So, in the future, I would like to keep this role as it gives me a more challenging experience to overcome. But for me being able to handle this, I want to make sure the programming tasks are split correctly into small tasks to easier follow the progress on each implementation as well as make certain of its progress by talking to the other programmer more closely.

\section{Attendance}
Being a self-proclaimed lead programmer could be energy-consuming. However, this is a weak reason to be absent during studio practice and stand-up meetings. This is a mistake of mine I consciously have been making over and over again. Luckily, our team is understanding and easy-going on such absence, however, it could be a crucial factor for problems that may occur in the future. I've been working rather hard for this project though not attending the studio practice and instead working from home. Nevertheless, I could've lost some trust in my teammates for this reason, which could potentially become critical in the second semester. I want to avoid that as much as possible as it gives me a strong experience in what a real, professional job in the industry will be like. 

As a lead programmer, this is crucial. Not only might I miss some important events during my absence, but also I could miss any changes to the sprint during an unplanned meeting of the team. This, in my opinion, is not acceptable if I want to become a good leader that my team trusts in and is proud of. For that, I simply want to make sure I attend every important meeting and be aware of any changes to the project by giving myself a motivation such as becoming a more reliable member of the team. Thus being able to make decisions with all our members depending on the current situation of our team. 

\section{Conclusion}
In conclusion, my main consideration is to become a reliable team member that could potentially guide the team in the right direction if something goes downhill. To achieve this I will have to get out of my comfort zone and learn such skills as aiding development process of other disciplines when necessary, communicating closely with designers, programmers, and animators, having a strong understanding of the current progress of the project and prioritizing tasks that have to be implemented during sprints. All of this will help me achieve my goal as well as becoming more confident in myself. I want to become a leader that John Quincy Adams described as - "If your actions inspire others to dream more, learn more, do more and become more, you are a leader".

\bibliographystyle{ieeetran}
\bibliography{references}
\nocite{*}

\end{document}
